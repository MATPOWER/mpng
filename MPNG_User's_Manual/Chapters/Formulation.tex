\chapter{Optimal Power and Natural Gas Flow}
\label{chap:formulation}

\section*{Nomenclature}
\subsection*{Indexes}
\begin{labeling}{Longer label\quad}
\item [$i$, $j$] Gas nodes.
\item [$m$, $n$] Electric nodes (buses). 
\item [$o$] Gas pipeline.
\item [$c$] Compressor.
\item [$l$] Transmission line.
\item [$w$] Gas well.
\item [$e$] Power generator.
\item [$ref$] Reference bus.
\item [$r$] Spinning reserve.
\item [$\sigma$] Type of gas load.
\end{labeling}

\subsection*{Parameters}

\begin{labeling}{Longer label\quad}
\item [$\alpha^{i}_{\pi_+}, \alpha^{i}_{\pi _-}$] Penalties for over-pressure and under-pressure at node $i$.
\item [$\alpha_{\gamma}$] Penalties for non-supplied gas.
\item [$\alpha_{\epsilon}$] Penalties for non-supplied electricity.
\item [$C^{w}_{G}$] Gas cost at the well $w$.
\item [$C^{oij}_{O}$] Transport cost of pipeline $o$, from node $i$ to node $j$.
\item [$C^{cij}_{C}$] Compression cost of compressor $c$, from node $i$ to node $j$.
\item [$C^{i}_{S}$] Storage cost at node $i$.
\item [$C^{i}_{S_+}$] Storage outflow price at node $i$.
\item [$C^{i}_{S_-}$] Storage inflow price at node $i$.
\item [$C^{e}_{E}$] Power cost generation (excluding gas cost).
\item [$\eta^{q}_{e}$] Thermal efficiency at generator $e$.
\item [$D_{g}^{i \sigma}$] Gas demand of type $\sigma$ at node $i$.
\item [$D_{e}^{tm}$] Electricity demand in the bus $m$ at time $t$.
\item [$\bar{g}^{w}$, $\underline{g}^{w}$] Gas production limits.
\item [$\overline{\pi}^{i}$, $\underline{\pi}^{i}$] Quadratic pressure limits at node $i$.
\item [$S^{i}_{0}$] Initial stored gas at node $i$.
\item [$\overline{S}^{i}$, $\underline{S}^{i}$] Storage limits at node $i$.
\item [$\kappa^{oij}$] Weymouth constant of pipeline $o$.
\item [$\delta^{oij}$] Threshold for gas flow capacities.
\item [$\beta^{cij}$] Compression ratio of compressor $c$.
\item [$Z^{c}$] Ratio parameter of compressor $c$.
\item [$B^{c}$] Compressor design parameter of compressor $c$.
\item [$x$, $y$, $z$] Gas consumption parameters of gas-driven compressors.
\item [$\overline{f}^{oij}_{g}$] Gas transport capacity of pipeline $o$, from node $i$ to node $j$.
\item [$\overline{f}^{cij}_{g}$] Gas flow capacity of compressor $c$, from node $i$ to node $j$.
\item [$\overline{f}^{i}_{s}$,$\underline{f}^{i}_{s}$] Storage outflow capacities at node $i$.
\item [$\overline{p}_{g}^{e}$, $\underline{p}_{g}^{e}$] Active power generation limits of generator $e$.
\item [$\overline{q}_{g}^{e}$, $\underline{q}_{g}^{e}$] Reactive power generation limits of generator $e$.
\item [$\overline{V}^{tm} \underline{V}^{tm}$] Voltage limits for bus $m$ at time $t$.
\item [$\mathbb{S}^{l}$] Transmission capacity of power line $l$.
%\item [$x^{mn}_{l}$] Reactance of line $l$, from bus $m$ to bus $n$.
\item [$R^{tr}$]  Spinning reserve in the $r$-th spinning reserve zone at time $t$.
%\item [$M$] Generators assignment matrix.
%\item [$L$] Compressors assignment matrix.
\item [$u^{te}$] Unit commitment state for generator $e$ at time $t$.
\item [$\tau^{t}$] Energy weight related to period of time $t$.
\item [$E^{e}$] Available energy for hydroelectric generator $e$, \break during the total analysis window.
\end{labeling}

\subsection*{Sets}

\begin{labeling}{Longer label\quad}
\item [$\cal{N}$] Gas nodes, $\left| \cal{N} \right|= n_{\cal{N}}$.
\item [$\cal{N}_{S}$] Gas nodes with storage, $\cal{N}_{S} \subset \cal{N} $, $\left| \cal{N}_{S} \right|= n_{\cal{S}}$.
\item [${\cal{O}}$] Gas pipelines, $\left| \cal{O}  \right| = n_{\cal{O}}$
\item [${\cal{C}}$] Compressors, ${\cal{C}}_{G} \cup {\cal{C}}_{E}$, $\left| \cal{C}  \right| = n_{\cal{C}}$ 
\item [${\cal{C}}_{G}$] Compressors controlled by natural gas, ${\cal{C}}_{G} \subseteq {\cal{C}}$, $\left| {\cal{C}}_{G}  \right| = n_{{\cal{C}}_{G}}$ 
\item [${\cal{C}}_{E}$] Compressors controlled by electric power, ${\cal{C}}_{E} \subseteq {\cal{C}}$, $\left| {\cal{C}}_{E}  \right| = n_{{\cal{C}}_{P}}$ 
\item [${\cal{W}}$] Gas wells, $\left| \cal{W} \right|= n_{\cal{W}}$.
%\item [${\cal{W}}^{i}$] Gas wells at node $i$, ${\cal{W}}^{i} \subset \cal{W} $, $\left| {\cal{W}}^{i} \right|= n_{{\cal{W}}^{i}}$.
\item [$\cal{B}$] Power buses, $\left| \cal{B} \right|= n_{\cal{B}}$.
\item [$\cal{L}$] Power lines, $\left| \cal{L} \right|= n_{\cal{L}}$.
\item [$\cal{E}$] Power unit generators, ${\cal{E}}_{H} \cup {\cal{E}}_{G}^{i} =\cal{E}$, $\left| \cal{E} \right|= n_{\cal{E}}$.
\item [${\cal{E}}_{H}$] Hydroelectric power units, ${\cal{E}}_{H} \subseteq \cal{E} $, $\left| {\cal{E}}_{H} \right|= n_{{\cal{E}}_{H}}$.
\item [${\cal{E}}^{i}_{G}$] Gas-fired power units connected to gas node $i$, \break ${\cal{E}}^{i}_{G} \subseteq \cal{E}$, $\left| {\cal{E}}^{i}_{G} \right|= n_{{\cal{E}}_{G}}$.
\item [${\cal{Z}}_{r}$] Spinning reserve zones. 
\item [${\cal{F}}^{i}_{G}$, ${\cal{T}}^{i}_{G}$] Connected pipelines to node $i$ at side \textit{From} or \textit{To}.
\item [${\cal{F}}^{i}_{C}$, ${\cal{T}}^{i}_{C}$] Connected compressors to node $i$ at side \textit{From} or \textit{To}.
\item [${\cal{F}}^{m}_{E}$, ${\cal{T}}^{m}_{E}$] Connected power lines to bus $m$ at side \textit{From} or \textit{To}. 
\item [$\cal{T}$] Periods of analysis.
\item [$\Sigma$] Different types of gas demands.
\end{labeling}

\subsection*{Variables}

\begin{labeling}{Longer label\quad}
\item [${f}_{g}^{oij}$] Gas flow in pipeline $o$, from node $i$ to node $j$.
\item [${f}_{g_+}^{oij}$ ${f}_{g_-}^{oij}$] Positive and negative gas flow in pipeline $o$.
\item [${f}_{g}^{cij}$] Gas flow in compressor $c$, from node $i$ to node $j$.
\item [$\psi^{c}$] Power consumed by compressor $c$.
\item [$\phi^{c}$] Gas consumed by compressor $c$, connected to node $i$ at side \textit{from}.
\item [$\gamma^{i \sigma}$] Non-served gas of type $\sigma$ at node $i$.
\item [$\pi^{i}$] Quadratic pressure at node $i$.
\item [${\pi}^{i}_{+}$, ${\pi}^{i}_{-}$] Quadratic over/under pressures at node $i$.
\item [$g^{w}$] Gas production at well $w$.
\item [$f_{s}^{i}$] Storage outflow difference.
\item [$f_{s_+}^{i}$, $f_{s_-}^{i}$] Storage outflow and inflow.
\item [$p_{g}^{te}$] Active power production of generator $e$ at time $t$.
\item [$q_{g}^{te}$] Reactive power production of generator $e$ at time $t$.
\item [$V^{tm}$] Voltage magnitude of bus $m$ at time $t$.
\item [$\theta^{tm}$] Voltage angle of bus $m$ at time $t$.
\item [$\epsilon^{tm}$] Non-served active power of bus $m$ at time $t$.
\end{labeling}
%\section{~}
\section{Objective function}

The cost function presented in Equation \ref{eq:obj_func} consists of several linear components, both from the power and the gas networks. It comprises the sum of the operation cost  of the interdependent system. In the case of the natural gas network, it includes the natural gas extraction cost, the transportation cost, the storage cost, the penalties associates with quadratic over/sub pressures, and non-supplied natural gas. In the case of the power network, it includes the generation cost and the penalties associates with non-supplied power demand.

%The cost function is represented by the equation \ref{obj_func} and is composed by several linear components. The first component is the gas production cost at each of the wells. As well as the first component, the second one is the power generation cost for every power plant, for the complete period. \color{red}The third component is the cost of the storage flow at every node with storage availability. \color{black}The fourth  expression is the storage cost, having into account the previous storage level and the outing flow. The fifth component is the gas transport cost for each pipeline. Finally, the last three component are the penalties cost for over/under pressure, non-supply gas and non-supply power, respectively. Related to the non-gas supply, is important to clarify that the non-supply cost depends on the type of user. 

\begin{equation}
\begin{aligned}
C \left( x \right) = & \sum_{w \in \cal{W}}{C^{w}_{G} g^{w}} + \sum_{t \in \cal{T}} {\tau}^{t}  \sum_{e \in \cal{E}} {C^{e}_{E} p_{g}^{te}}\\ 
				& + \sum_{i \in {\cal{N}}_{S}}{\left({C^{i}_{S_+} f^{i}_{s_+}} - {C^{i}_{S_-} f^{i}_{s_-}}  \right)}\\
				& + \sum_{i \in {\cal{N}}_{S}}{C^{i}_{S} \left( S^{i}_{0} - f^{i}_{s} \right)} \\
				& + \sum_{o \in \cal{O}}{C^{oij}_{O} f^{oij}_{g_+}} - \sum_{o \in \cal{O}}{C^{oij}_{O} f^{oij}_{g_-}} \\
				& + \sum_{c \in \cal{C}}{C^{cij}_{C} f^{cij}_{g}} \\ 
				& + \sum_{i \in \cal{N}}{\alpha^{i}_{\pi_+} \pi^{i}_{+}} + \sum_{i \in \cal{N}}{ \alpha^{i}_{\pi_-} \pi^{i}_{-}} \\
				& + \sum_{i \in \cal{N}}\sum_{\sigma \in \Sigma} {\alpha_{\gamma}^{i \sigma}\gamma^{i \sigma}} + \alpha_{\epsilon} \sum_{t \in \cal{T}} {\tau}^{t} \sum_{m \in \cal{B}} {\epsilon^{tm}}  
\end{aligned}
\label{eq:obj_func}
\vspace{0.3cm}
\end{equation}

The aim of the Optimal Power and Natural Gas Flow is to minimize the linear objective $C(x)$ subject the constraints explained below.

\section{Constraints}

%The model of the optimal power and natural gas flow consists in maintain the balance in both networks. For the natural gas network the nodal balance must be has linear and non-linear constraints, which model the 

\subsection{Gas network}

Equation \ref{eq:node_gas_balance} shows the gas balance for a specific node $k$ during a day. This gas balance is composed by the incoming and outgoing flows associated with pipelines and compressors at the node $k$, the outgoing stored flow in the available storage, the generation related to that node, and the total gas demand. In detail, the gas demand is composed by the required flow in the gas-fired power plants and compressors, and the total gas demand of the rest of the consumers, excluding the non-supplied natural gas. 

\begin{equation}
\begin{array}{l}\displaystyle
\sum_{o \in {\cal{T}}^{k}_{G}}{{f}_{g}^{oij}} - \sum_{o \in {\cal{F}}^{k}_{G}}{{f}_{g}^{oij}} + \sum_{c \in {\cal{T}}^{k}_{C}}{{f}_{g}^{cij}} - \sum_{c \in {\cal{F}}^{k}_{C}}{ \left({f}_{g}^{cij} + \phi^{c}\right)} + {f}^{k}_{s}\\ [0.8cm] \displaystyle
+ \sum_{w \in {\cal{W}}^{k}}{g^{w}}  - \sum_{t \in \cal{T}} {\tau}^{t} \sum_{e \in {\cal{E}}^{k}_{G}}\left({\eta^{q}_{e}}\cdot {p_{g}^{te}}\right) = {\sum_{\sigma \in \Sigma}\left( D^{\sigma k}_{g} - \gamma^{\sigma k}\right)} \\
\end{array}
,\;\; \forall k \in \cal{N}.
\label{eq:node_gas_balance}
\end{equation}

\newpage
\noindent \textbf{Nodes}\\

Constraints related to node $k$ are those that involve variables of non-supplied gas demands and quadratic pressures. The non-supplied gas at node $k$ for a specific user $\sigma$ can not exceed the total demand of that user. This constraint is presented in Equation \ref{eq:nsg_limits}.

\begin{equation}
{0 \le \gamma^{\sigma k} \le D^{\sigma k}_{g}, \quad \forall \sigma \in \Sigma, \quad \forall k \in \cal{N}}.
\label{eq:nsg_limits}
\vspace{0.3cm}
\end{equation}

Moreover, Equations \ref{overp} and \ref{underp} are the constraints that characterize the quadratic over-pressure and under-pressure at each node of the system, respectively. 

\begin{equation}
\begin{array}{l}
 \pi^{k} \le \overline{\pi}^{k} + \pi^{k}_{+}\\
 0 \le \pi^{k}_{+}\\
\end{array} 
,\quad \forall k  \in {\cal{N}}\\ 
\label{overp}
\end{equation}

\begin{equation}
\begin{array}{l}
\underline{\pi}^{k} - \pi^{k}_{-} \le \pi^{k}\\
0 \le \pi^{k}_{-}\\
\end{array} 
,\quad \forall k  \in {\cal{N}}.\\ 
\label{underp}
\vspace{0.3cm}
\end{equation}

\noindent \textbf{Wells}\\

The constraints related to the gas wells injection depends on each well characteristics. The injection limits are represented as follows:

\begin{equation}
\underline{g}^{w} \le g^{w} \le \overline{g}^{w}, \quad \forall w \in \cal{W}.
\label{eq:g_limits}
\end{equation}
%\vspace{5cm}

\noindent \textbf{Pipelines} \\

The gas flow in pipeline $o$, connecting nodes $i$ and $j$, depends on the quadratic pressures of such nodes. This behavior is given by the Weymouth Equation \ref{eq:wey}. In particular, the gas flow is allowed to be bidirectional within a physical limit for a maximum daily transportation according to Equation \ref{fgo_limits}. On the other hand, as the transport cost is always a positive quantity no matter the direction, variables $f^{oij}_{g+}$ and $f^{oij}_{g-}$ support a positive contribution to the objective function. Equation \ref{eq:fgo} shows the sum of both directional flows to determine the actual flow in the direction $from$  node $i$ - $to$ node $j$. These flow variables are constrained by Equations \ref{eq:fgopos_limits} and \ref{eq:fgoneg_limits}. As seen, the positive gas flow must be greater or equal than zero but lower or equal than the maximum transport capacity multiplied by a threshold factor, $\delta^{oij}$, which states an extra flow margin. Analogously, the negative gas flow has similar bounds in the negative side.


\begin{equation}
{f}^{oij}_{g} = {{\kappa}^{oij}} sgn \left(\pi^{i}-\pi^{j}\right) {\sqrt{\left|\pi^{i}-\pi^{j}\right|}}, \quad \forall o \in {\cal{O}}
\label{eq:wey}
\end{equation}
\begin{equation}
 - \overline{f}^{oij}_{g}  \le f^{oij}_{g} \le  \overline{f}^{oij}_{g},  \quad \forall o \in {\cal{O}}
\label{fgo_limits}
\end{equation}
\begin{equation}
{f}^{oij}_{g} =  f^{oij}_{g_+} + {f}^{oij}_{g_-}, \quad \forall o \in {\cal{O}}
\label{eq:fgo}
\end{equation}
\begin{equation}
0 \le f^{oij}_{g_+} \le \delta^{oij} \cdot \overline{f}^{oij}_{g}, \quad \forall o \in {\cal{O}}
\label{eq:fgopos_limits}
\end{equation}
\begin{equation}
- \delta^{oij} \cdot \overline{f}^{oij}_{g} \le f^{oij}_{g_-} \le 0, \quad \forall o \in {\cal{O}}.
\label{eq:fgoneg_limits}
\end{equation}

\noindent \textbf{Compressors}\\

Compressors allow to recover pressure losses through the gas network. This process demands energy. The power consumption of compressor $c$ between the suction node $i$ and the discharge node $j$ is given by Equation \ref{eq:hp_fc}. It depends on the quadratic pressure ratio of nodes $i$ and $j$, and the gas flow through the compressor. Moreover, the additional gas required by a gas-driven compressor relies on its power consumed as shown in Equation \ref{eq:g_fc}. As the gas flow through compressors is restricted to flow in one direction, the flow limits of compressor $c$ are fixed according to Equation \ref{eq:gfa_limits}. Finally, the quadratic pressure at suction and discharge nodes must fall within acceptable margins, as presented in Equation \ref{eq:presa_rel}, where $\beta^{cij}$ is the maximum compressor ratio.\\

\begin{equation}
\psi^{c} = {B^{c}}{f}^{cij}_{g} \cdot \left[ {\left( \frac{\pi^{j}}{\pi^{i}} \right)}^{\frac{Z^{c}}{2}} - 1 \right],   \quad \forall c \in {\cal{C}}
\label{eq:hp_fc}
\end{equation}
\begin{equation} 
{\phi}^{c} = x + y \psi^{c} +  z {\psi^{c}}^{2},  \quad \forall c \in {\cal{C}}_{G}
\label{eq:g_fc}
\end{equation}
\begin{equation}
0 \le {f}^{cij}_{g} \le \overline{f}^{cij}_{g},  \quad \forall c \in {\cal{C}}
\label{eq:gfa_limits}
\end{equation}
\begin{equation}
\begin{aligned}
&\pi^{i} \le \pi^{j} \le \beta^{cij} \pi^{i}\\
&\beta^{cij} \ge 1 \\
\end{aligned}
,\quad\forall i,j \in {\cal{N}}, \quad \forall c \in {\cal{C}}.
\label{eq:presa_rel}
\end{equation}

\noindent \textbf{Storage}\\

The storage outflow difference is the subtraction between the storage outflow and the storage inflow at the storage nodes; this relationship is represented by Equation \ref{eq:fs}. Additionally, Equation \ref{eq:fs_limits} shows that the outflow storage difference is restricted by the maximum and minimum amount of gas that is allowed to be injected to or demanded from the network in every storage node. Furthermore, as the storage unit can operate either as an injection or a demand for the network, Equations \ref{eq:fs+} and \ref{eq:fs-} represent the possible storage unit behavior. In particular, the maximum gas amount that can be injected is the difference between the currently available stored gas and the minimum possible volume of the unit. Similarly, the maximum inflow is the difference between the maximum volume of the unit and the currently available stored gas.

\begin{equation}
f_{s}^{k} = f_{s_+}^{k}  - f_{s_-}^{k}, \quad \forall k \in \cal{N}
\label{eq:fs}
\end{equation}
\begin{equation}
\underline{f}^{k}_{s} \le f^{k}_{s} \le \overline{f}^{k}_{s}, \quad \forall k \in \cal{N}
\label{eq:fs_limits}
\end{equation}
\begin{equation}
0 \le f_{s_+}^{i} \le S^{k}_{0}  - \underline{S}^{k}, \quad \forall k \in \cal{N}
\label{eq:fs+}
\end{equation}
\begin{equation}
0 \le f_{s_-}^{i} \le \overline{S}^{k} - S^{k}_{0}, \quad \forall k \in \cal{N}.
\label{eq:fs-}
\end{equation}

\subsection{Power network}

The power network balance equations of active and reactive power are given by Equation \ref{eq:power_balance}. The model also takes into consideration the non-supplied power demand and the power consumed by compressors connected to the power system.
 
\begin{equation}
\begin{array}{l}
g_{p_m}\left(\theta^{tm}, V^{tm}, p_{g}^{te}, \epsilon^{te}, \psi^{c}\right) = 0\\ 
g_{q_m}\left(\theta^{tm}, V^{tm}, q_{g}^{te}\right) = 0\\
\\
\quad \forall m \in {\cal{B}} \quad \forall t  \in {\cal{T}} \quad \forall c  \in {\cal{C}}_{E}. 
\end{array}
\label{eq:power_balance}
\vspace{0.3cm}
\end{equation}

The main variables of the power system are the voltage angles $\theta^{tm}$ and the voltage magnitudes $V^{tm}$ at each bus $m$ for every period of time $t$, as well as the active generation $p^{te}_{g}$ and reactive generation $g^{te}_{g}$ at each generator $e$ for each time period. The voltage limits are represented by Equation \ref{eq:v_lims}, and the generation limits are shown in Equation \ref{eq:pq_lims}.

\begin{equation}
\begin{aligned}
&\theta^{t_\text{ref}} = 0\\
&\underline{V}^{tm} \le V^{tm}  \le \overline{V}^{tm}\\
\end{aligned} 
\;,\quad \forall m \in {\cal{B}} \quad \forall t  \in {\cal{T}}  
\label{eq:v_lims}
\end{equation}

\begin{equation}
\begin{aligned}
&\underline{p}_{g}^{e} \le p_{g}^{te}  \le \overline{p}_{g}^{e}\\
&\underline{q}_{g}^{e} \le q_{g}^{te}  \le \overline{q}_{g}^{e}\\
\end{aligned} 
\;,\quad \forall e \in {\cal{E}}, \quad \forall t  \in {\cal{T}}.  
\label{eq:pq_lims}
\vspace{0.3cm}
\end{equation}

The power flow limits are bidirectional and are presented in Equation \ref{eq:sft_lims}, where $\mathbb{S}_{fl}$ and $\mathbb{S}_{tl}$ are the power injections at side $from$ and $to$ of line $l$, respectively.

\begin{equation}
\begin{aligned}
&|\mathbb{S}_{fl}\left(\theta,V\right)| \le \overline{\mathbb{S}}_{fl}\\
&|\mathbb{S}_{tl}\left(\theta,V\right)| \le \overline{\mathbb{S}}_{tl}\\
\end{aligned} 
\;,\quad \forall l \in {\cal{L}}.
\label{eq:sft_lims}
\vspace{0.3cm}
\end{equation}

 The non-supplied active power demand at bus $m$ can not exceed the total bus demand, according to Equation \ref{nsd_limits}.
 
\begin{equation}
0 \le \epsilon^{tm} \le D^{tm}_{e}, \quad \forall m \in {\cal{B}}, \quad \forall t  \in {\cal{T}}.  
\label{nsd_limits}
\vspace{0.3cm}
\end{equation}

The model also considers the required spinning reserve for each zone $r$ at every time $t$. This constraint is given by Equation \ref{eq:reserve}.

\begin{equation}
%\begin{array}{l}
\sum_{e \in {\cal{Z}}_{r}}{u^{te} \left( \overline{p}_{g}^{e} - p_{g}^{te} \right)} \ge R^{tr} ,\quad \forall r \in {\cal{Z}}_{r}, \quad \forall t  \in {\cal{T}}.
%\end{array}
\label{eq:reserve}
\vspace{0.3cm}
\end{equation}

Finally, the model takes into consideration the maximum available energy during a day for certain generators, especially the energy stored in the dams for hydro-power plants. Equation \ref{eq:hydro_e} represents such a constraint.

\begin{equation}
\sum_{t \in {\cal{T}}}{{\tau}^{t} p_{g}^{te}} \le E^{e}, \quad \forall e \in {\cal{E}}_{H}
\label{eq:hydro_e}.
\vspace{0.3cm}
\end{equation}

\newpage
\chapter{Fundamentals of Natural Gas Flow}
\label{chap:fund_NGF}

The steady-state Natural Gas Flow (NGF) problem for transmission networks aims to find the value for a set of state-variables that satisfy the flow balance in all nodes. We show how the NGF can be derived in a similar way as the Power Flow (PF) problem is introduced for power systems. In  particular, a set of nonlinear equations must be solved where the definition of the state-variables depends on the selected models for all the elements of the system. In this section, we derive the NGF problem and introduce the modeling for the main elements considered in \mpng{}: nodes, wells, pipelines, compressors, and storage units.

\section{Modeling}
\label{sec:gas_modeling}

An exact description of the natural gas flow in transmission networks requires applying the laws of fluid mechanics and thermodynamics []. Complex analyzes provide an accurate description for variables such as temperature, pressure, flow, adiabatic head, among others, for all time instants. However, as the primary concern of \matpower{} (and so does \mpng{}) is the system operation in steady-state, we define some models to describe the main elements of the default natural gas network, as explained below. 

\subsection{Nodes}
\label{subsec:nodes}

By definition, a node is the location of a natural gas system where one or more elements are connected. Users are commonly associated with a node where a stratified demand is modeled as different market segments that get different priorities. Figure [] shows the $i$-th node of a gas network with some traditional markets connected to form the nodal demand $f_{dem}$. The primary variable for a node is pressure $p_i$.


\subsection{Wells}
\label{subsec:wells}

\subsection{Pipelines}
\label{subsec:pipelines}

\subsection{Compressors}
\label{subsec:compressors}

\subsection{Storage Units}
\label{subsec:sto_units}



\section{Derivation of the Natural Gas Flow Problem}
\label{sec:NGF_problem}


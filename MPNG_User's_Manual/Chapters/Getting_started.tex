\chapter{Getting started}
\label{chap:get_started}

\section{System Requirements}
\label{sec:requirements}

To use \mpng{} you will need the following system requirements:

\begin{itemize}
	\item[\checkmark] \matlab{}\textsuperscript{\tiny \textregistered} version 7.3 (R2016b) or later.\footnote{\matlab{} is available from The MathWorks, Inc. (\url{https://www.mathworks.com/}). An R2016b or later \matlab{} version is required as the \mpng{} code uses \matlab{}-files with multiple function declarations.}
	
	\item[\checkmark] \matpower{} version 7.0 or later.\footnote{\matpower{} is available thanks to the Power Systems Engineering Research Center (\pserc) (\url{https://matpower.org})}
\end{itemize}

\section{Getting \mpng{}}
\label{sec:get_mpng}

You can obtain the \emph{current development version} from the \matpower{} Github repository: \url{https://github.com/MATPOWER/mpng.git}.


\section{Running a Simulation}
\label{sec:simulate}

The primary functionality of \mpng{} is to solve optimal power and natural gas flow problems. Running a simulation using \mpng{} requires (1) preparing the natural gas input data, (2) specifying the interconnection input data to couple the gas network to the power system, (3) invoking the function to run the integrated simulation and (4) accessing and viewing the results.\\

The classical \matpower{} input data is a ``\matpower{}-case'' struct denoted by the variable \code{mpc}\cite{matpower_manual}. To integrate the power and natural gas systems we use the extended Optimal Power Flow (OPF) capability of \matpower{}. Namely, we model the natural gas system and its connection to the power system via general user nonlinear constrains. Then, \mpng{} uses an extended ``\matpower{}-gas case'' struct denoted by the variable \code{mpgc}. In particular, \code{mpgc} is a traditional \matpower{}-case struct with two additional fields, \code{mpgc.mgc} and \code{mpgc.connect} standing for the natural gas case and interconnection case, respectively.

\subsection{Preparing the Natural Gas Case}
\label{subsec:gas_case}

The input data of the natural gas system are specified in a set of matrices arranged in a \matlab{} struct that we refere to as the ``gas case'' (\code{mpgc.mgc}). The structure of such a gas case is formatted in a similar way to the \matpower{}-case but holding the natural gas information that comprise gas bases, nodes, wells, pipelines, compressors, and storage units. See Appendix~\ref{app:gas_format} for more details about the gas case structure.

\subsection{Connecting the Gas Case to the \matpower{} Case}
\label{subsec:connect_case}

The input data regarding the connection between the power and natural gas systems are declared in a set of matrices packaged as a \matlab{} struct which we call ``interconnection case'' (\code{mpgc.connect}). The structure of this case contains specific information about coupling elements as gas-fired power generators and power-and-gas-driven compressors, according to the optimization model described in section~\ref{chap:formulation}. See appendix~\ref{app:connect_format} for more details about the interconnection case structure.

\subsection{Solving the Optimal Power-Gas Flow}
\label{subsec:solve_OPGF}

Once the \matpower{}-gas case is properly formatted, one can invoke the solver using the (mandatory) \code{mpgc} struct and the traditional (optional) \matpower{} options struct \code{mpopt}. The calling syntax at the \matlab{} prompt is as follows:  

\begin{Code}	
>> results = mpng(mpgc,mpopt);
\end{Code}

For more details, type:

\begin{Code}
>> help mpng
\end{Code}





